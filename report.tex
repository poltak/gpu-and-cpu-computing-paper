\documentclass[a4paper,11pt]{article}

\usepackage[hidelinks]{hyperref}
\usepackage[font={small}]{caption}
\usepackage{poltakmacros}           % Personal macros included in file 'poltakmacros.sty'.
\usepackage{geometry}

\geometry{margin=2cm}


\author{Jonathan Poltak Samosir}
\title{FIT3143 Assignment 2}


\begin{document}

\maketitle
\thispagestyle{empty}

\begin{abstract}
\end{abstract}
\smallskip
\noindent \textbf{Keywords.} blah, blah, blah
\newpage
\pagenumbering{roman}
\tableofcontents
\newpage

\pagenumbering{arabic}

% TODO: Split sections into different files
\section{Introduction} % (fold)
\label{sec:introduction}

% section introduction (end)


\section{Introduction to the multi-core CPU} % (fold)
\label{sec:introduction_to_the_multi_core_cpu}

\subsection{The central processing unit} % (fold)
\label{sub:the_central_processing_unit}
The central processing unit --- hereby referred to as the CPU --- is a fundamental piece of hardware within modern computers that handles
the fetching, decoding and execution of each instruction of a computer program~\cite{web:CPUWiki}. It handles the processing
of these instructions through the use of two key components: the arithmetic logic unit --- hereby referred to as the ALU
--- for basic arithmetical and logical operations, and the control unit --- hereby referred to as the CU --- for the input/output
operations between the CPU and memory, along with the overall control of the fetch, decode and execute cycle. Given the
physical hardware constraints, only a single instruction may be processed at any given time by a CPU.
% subsection the_central_processing_unit (end)

\subsection{The multi-core CPU} % (fold)
\label{sub:the_multi_core_cpu}
To define the multi-core CPU, it is perhaps important to first define the term ``core''. Like many terms in the area of computing,
the term ``core'' is often misused or used with many different meanings attached to it. For the purposes of this paper, the
term ``core'' simply refers to actual CPUs as defined in~\sectref{sub:the_central_processing_unit}.

Given this definition of a core, the multi-core CPU can simply be defined as a piece of computing hardware that contains
two or more cores. As the multi-core CPU contains more than one core, rather than being constrained to processing a single
instruction at any given time, in theory it is possible to be processing $n$ instructions on an $n$-core CPU at any given time.

Given what multi-core CPU hardware makes possible, since their introduction to consumers in the early 2000s, the multi-core
CPU has changed the way programmers design their programs to take advantage of hardware. Programs that are designed with
code concurrency in-mind now have advantages that were only ever available previously on systems that have multiple physical
CPU chips.
% subsection the_multi_core_cpu (end)

\subsection{Multi-core CPU architecture} % (fold)
\label{sub:multi_core_cpu_architecture}
The most used multi-core CPU micro-architecture families in consumer computing today are arguably Intel's ``Core'' line
and AMD's

% subsection multi_core_cpu_architecture (end)


% section introduction_to_the_multi_core_cpu (end)


\section{Programming the multi-core CPU} % (fold)
\label{sec:programming_the_multi_core_cpu}

% section programming_the_multi_core_cpu (end)


\section{Introduction to the GPU} % (fold)
\label{sec:introduction_to_the_gpu}

% section introduction_to_the_gpu (end)


\section{Differences in programming the GPU} % (fold)
\label{sec:differences_in_programming_the_gpu}

% section differences_in_programming_the_gpu (end)


\section{Parallelism in the multi-core CPU and GPU} % (fold)
\label{sec:parallelism_in_the_multi_core_cpu_and_gpu}

% section parallelism_in_the_multi_core_cpu_and_gpu (end)


\section{Conclusion} % (fold)
\label{sec:conclusion}

% section conclusion (end)


\bibliographystyle{acm}
\bibliography{report}

\end{document}
